\ifdefined\BadgeCSV
\else
	\newcommand{\BadgeCSV}{badges.D.csv}
\fi


\renewcommand{\ticketdefault}{}%


\newcommand{\striked}[3]{
	\ifthenelse{\equal{#3}{#2}}{
		\sout{#1}	
	}{
		#1 #2	
	}
}

\newcommand{\badgesichtbar}[4]{\ticket{%
% das fossgis2023_background.pdf liegt in A6 quer vor %
% und kann so komplett als Hintergrund Bild genutzt werden %

% Breite DIN A 7 74mm x 105mm
\put(0,0){\includegraphics[width=105mm]{../imgs/2024/fossgis_background.pdf}}
\put(5mm,30mm){
  % in der \parbox werden lange Namen umgebrochen
  % \centering durch \raggedleft oder \raggedright erstetze um
  % zu prüfen ob die Seitenränder eingehalten werden

  \parbox{93mm}{ \vfill \raggedleft \fontsize{32}{45}
  \textbf{#1}
  %\textbf{
  %\begin{Form}
  %\textbf{\TextField[multiline,value=foobar,width=10cm]{}}
  %\end{Form}
  %}
  \normalsize \\ {#4}
  		}

  }

\put(85mm,60mm){\parbox{10mm}{\raggedleft \footnotesize
		\ifthenelse{\equal{#2}{True}}{AV\\}{ }
		\ifthenelse{\equal{#3}{True}}{OS\\}{ }

}}
}}



\newcommand{\badgeinnenlinks}[8]{\ticket{
\put(8,45){
  % in der \parbox werden auch lange namen umgebrochen
  \parbox{8.5cm}{
   % dezente Farbe, damit diese infos nicht auf die Vorderseite durchscheinen
    \color{gray} \raggedright \small
    
    %%Nachname,Vorname,Ticket,TShirt,TBand,TListe,AV,Mittag,WS,EX,OSM
	Order: #1 \\  	
  	Name: #2 \\
  	Mail: #3 \\
  	Ticket: #4 \\  	
  	%Tagungsband: \ifthenelse{\equal{#5}{True}}{Ja}{---------} \\
  	\striked{Tagungsband:}{ #5}{False} \\
  	\striked{T-Shirt:}{ #6}{} \\
  	\striked{Essen:}{ #7}{} \\
  	Anmerkungen: #8
%  	T-Shirt: \ifthenelse{\equal{#6}{}}{---------}{#6} \\
%  	Essen: \ifthenelse{\equal{#7}{}}{---------}{#7} \\
  }
  }
}}

\newcommand{\badgeinnenrechts}[2]{\ticket{%
\put(2,50){
  % in der \parbox werden auch lange namen umgebrochen
  \parbox{90mm}{
   % dezente Farbe, damit diese infos nicht auf die Vorderseite durchscheinen
    \color{gray} \raggedright \small
    %%Nachname,Vorname,Ticket,TShirt,TBand,TListe,AV,Mittag,WS,EX,OSM
    Workshops: #1 \\ 
  	Exkursionen: #2 \\
  	%\begin{itemize}  \item Orchestrierung einer GDI über Docker \item Mapbender - Einstieg in den Aufbau von WebGIS-Anwendungen \item React basierte WebGIS Clients mit MapComponents \& MapLibre-gl \end{itemize}
  	%\begin{itemize} \item item1 \item 2\end{itemize}
  }
  }
}}
