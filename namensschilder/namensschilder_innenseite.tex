
\documentclass[a4paper, 12pt]{letter}
\usepackage[total={210mm,297mm},top=0mm,left=0mm,includefoot]{geometry}
\usepackage[schilder,rowmode]{ticket}
\usepackage{graphicx,palatino}
\usepackage{xcolor}
\usepackage[utf8]{inputenc} 
\usepackage{datatool} 
\usepackage{ifthen}

\usepackage[sfdefault]{ClearSans} %% option 'sfdefault' activates Clear Sans as the default text font
\usepackage[T1]{fontenc}


 %\overfullrule 10pt
 \hyphenpenalty 1000
 \exhyphenpenalty 10000

%\renewcommand*\rmdefault{iwona}

\renewcommand{\ticketdefault}{}%
\makeatletter
\@boxedtrue % Rahmen um Namensschild
\@emptycrossmarktrue % Falzmarken
\@cutmarktrue % Schnittmarken
\makeatother

\newcommand{\schildinnenlinks}[6]{\ticket{
\put(8,50){
  % in der \parbox werden auch lange namen umgebrochen
  \parbox{8.5cm}{
   % dezente Farbe, damit diese infos nicht auf die Vorderseite durchscheinen
    \color{gray} 
    %%Nachname,Vorname,Ticket,TShirt,TBand,TListe,AV,Mittag,WS,EX,OSM
	Order: #1 \\  	
  	Name: #2 \\
  	Mail: #3 \\
  	Ticket: #4 \\  	
  	T-Shirt: #5 \\
  	Essen: #6
  }
  }
}}

\newcommand{\schildinnenrechts}[2]{\ticket{%
\put(8,50){
  % in der \parbox werden auch lange namen umgebrochen
  \parbox{8.5cm}{
   % dezente Farbe, damit diese infos nicht auf die Vorderseite durchscheinen
    \color{gray} 
    %%Nachname,Vorname,Ticket,TShirt,TBand,TListe,AV,Mittag,WS,EX,OSM
    Workshops: #1 \\  	
  	Exkursionen: #2
  }
  }
}}

\begin{document} 
\DTLsetseparator{;}
\DTLloaddb{CSV}{bin/badges.csv}
\DTLsort{nachname}{CSV}

% order;name;nachname;mail;ticket;tl_name;tl_veroeff;tl_erhalten;
% essen;tshirt;av;tb;tb_adresse;osm_samstag;osm_name;exkursionen;workshops
\DTLforeach{CSV}{\order=order, \name=name,\type=ticket,\tshirt=tshirt,\mail=mail,
\av=av,\essen=essen,\ws=workshops, \ex=exkursionen}
%\DTLforeach{CSV}{\person=Name,\nickname=268184}
{
 %%Nachname,Vorname,Ticket,TShirt,TBand,TListe,AV,Mittag,WS,EX,OSM
  \schildinnenlinks{\order}{\name}{\mail}{\type}{\tshirt}{\essen}
  \schildinnenrechts{\ws}{\ex}
  %\ticket{}
} 
\end{document} 
